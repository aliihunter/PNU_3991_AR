\documentclass{book} 
\usepackage[top=3cm,right=3.5cm,bottom=3cm,left=3cm]{geometry}
\renewcommand{\baselinestretch}{1.7}
\usepackage[utf8]{inputenc}
\usepackage[english]{babel}
\usepackage{sectsty}
\sectionfont{
    \usefont{OT1}{phv}{b}{n}
    \sectionrule{0pt}{0pt}{-5pt}{1pt}}
\usepackage{wrapfig,lipsum}
\usepackage{graphicx}
\setlength{\parindent}{4em}
\setlength{\parskip}{1em}
\newcommand{\NewPart}[1]{\section*{\uppercase{#1}}}
\begin{document}
\NewPart{ Acknowledgements}{}
At starting of my career as a lecturer I was offered two subjects. One of them is automata theory. At the
very beginning I used to prepare notes, solve examples, etc., on plain paper. Those notes helped me a
lot in teaching this subject.\par
November 2008 steered my career to a new direction. On 3 November 2008, Professor C. T. Bhunia
left our institute. He used to inspire me in good teaching, research and in different creative activities.
His departure from our institute made me roofl ess. I got a huge time beyond my academic activities.
I started making hand written notes in soft copies which started the journey of writing the book. In
January 2012 the Express Learning book on Automata was published by Pearson Education. Since then
there was a desire of writing a text book on this subject which can cover the syllabuses of most of the
universities of India.\par
Through this work I want to pay respect to Professor C. T. Bhunia, Director NIT Arunachal, whose
blessings are always with me and I consider him as my Guru in the fi eld of technical education.
I want to pay respect to Bibhas Ch. Dhara, Head of the Department, IT, Jadavpur University and my
Ph.D. guide. He wanted all things to be perfect. He pointed out some mistakes in the Express Learning
book as well as some useful suggestions on Automata theory to make it a quality book.
My friend Bidesh Chakrabory and Arindam Giri, Assistant Professor, Computer Science helped
me a lot by providing needful information and consistent inspiration and support. A number of my
students also helped me a lot in selecting problems, solving problems, proof checking, etc. Among them
Satyakam Shailesh, Akash Ranjan, Nishant Kumar, Nijhar Bera and Somnath Kayal deserve special
mention.\par
Finally I want to express gratitude to my colleagues especially Mrinmoy Sen, Sabyasachi Samanta,
Ashish Bera, Anupam Pattanayak, Milan Bera, Mrityunjay Maity and Tanuka Sinha, Bapida, who are
an important part of our department.\par
I express my thanks to parents Arun Kumar Kandar and Hirabati Kandar for their constant support
and encouragement during the preparation of this book.\par
Neha Goomer of Pearson Education deserves a special mention. My sincere thanks to Neha for
making my dream of publishing the book come true.\par
\begin{minipage}{0.8\textwidth}\raggedleft
Shyamalendu Kandar
\end{minipage}
\newpage
\NewPart{About the Author}{}

\begin{wrapfigure}[9]{r}{0.2\textwidth}
    \includegraphics[width=0.8\textwidth]{pic.jpg}
\end{wrapfigure}Shyamalendu Kandar is working as an Assistant Professor of Computer
Science and Engineering department of Haldia Institute of Technology.
He served as a co-ordinator of M.Tech (IT) of HIT (centre distance mode)
with Javadpur University. Professor Kandar has done his M.Tech in IT
from Jadavpur University in 2006. He has a number of research papers
in reputed international conferences and journals. His subject of interests
are Automata Theory, Compiler Designing, Algorithm, Object oriented
programming, Web Technology, etc. His areas of research interest are
cryptography, secret sharing, Image processing, etc.
\end{document}