\documentclass{book} 
\usepackage[top=3cm,right=3.5cm,bottom=3cm,left=3cm]{geometry}
\renewcommand{\baselinestretch}{1.7}
\usepackage[utf8]{inputenc}
\usepackage[english]{babel}
  \newcommand{\titulo}[1]
    {
    \par\medskip
    \begin{noindent}
    \rule{\linewidth}{0.8pt}
    \fontsize{14pt}{0pt}{\selectfont\textbf{#1\\}}
    \end{noindent}
    }
\begin{document}
PAGE 8\\
WHEN IS CONSENT NEEDED? THE PUBLIC VERSUS\\
 PRIVATE DILEMMA \quad\quad 68\\
REDUCING THE POTENTIAL TO HARM \quad\quad 69\\
TIPS FOR ETHICAL e-RESEARCH \quad\quad 70\\
SUMMARY \quad\quad 71\\
REFERENCES \quad\quad 71
\titulo{CHAPTER SIX}
Collaborative e-Research \quad\quad 73 \\
TYPES OF e-RFSEARCH COLLABORATION \quad\quad 74\\ 
CHALLENGES OF e-RESEARCH COLLABORATION \quad\quad 76 \\
APPLICATIONS OF COLLABORATIVE SOFTWARE BY e-RESEARCHERS\quad\quad 77 \\
MICROSOFI"S SHAREPOINT"' TEAM SERVICES \quad\quad 78 \\
COMMUNITYZERO\quad\quad 80 \\
GROOVE NVINORKS: PEER-TO-PEER COLLABORATION SOFTWARE \quad\quad 81 \\
TIME TRACKING \quad\quad 82 \\
APPLICATION OF COLLABORATIVE SOFTWARE BY e-RESEARCHERS\quad\quad 83\\
REFERENCES\quad\quad 84
\titulo{CHAPTER SEVEN}
 Semi-Structured and Unstructured Interviews \quad\quad 85 \\
UNSTRUCTURED VERSUS SEMI-STRUCTURED INTERVIEWS \quad\quad 86 \\
INTERVIEWING SKILLS\quad\quad 87 \\
INITIATING THE PROCESS\quad\quad 91 \\
ASKING THE QUESTIONS\quad\quad 94 \\
ANALYZING THE DATA\quad\quad 99 \\
TIPS FOR CONDUCTING INTERVIEWS \quad\quad 100 \\
SUMMARY \quad\quad 100 \\
REFERENCES\quad\quad 101
\newpage
PAGE 9
\titulo{CHAPTER EIGHT}
Focus Groups\quad\quad 102 \\
THE DIFFERENT KINDS OF NET-BASED FOCUS GROUPS\quad\quad 103 \\
ADVANTAGES AND DISADVANTAGES OF FACE-TO-FACE VERSUS\\
NET-BASED FOCUS GROUPS\quad\quad 105 \\
THE PROCESS\quad\quad 107 \\
GROUP SIZE\quad\quad 107 \\
ADVANTAGES AND DISADVANTAGES OF TEXT-BASED SYNCHRONOUS\\
VERSUS TEXT-BASED ASYNCHRONOUS FOCUS GROUPS\quad\quad 108 \\
PARTICIPANT CHARACTERISTICS\quad\quad 109 \\
ORGANIZATION\quad\quad 110 \\
THE MODERATOR\quad\quad 115 \\
BRINGING CLOSURE\quad\quad 117 \\
ANALYZING THE DATA\quad\quad 117 \\
TIPS FOR FACILITATING A SUCCESSFUL NET-BASED FOCUS GROUP\quad\quad 118 \\
SUMMARY\quad\quad 119 \\
REFERENCES\quad\quad 119
\titulo{CHAPTER NINE} 
Net-Based Consensus Techniques\quad\quad 120 \\
ADVANTAGES OF CONSENSUS TECHNIQUES\quad\quad 122\\
TYPES OF CONSENSUS TECHNIQUES\quad\quad 124 \\
DELPHI METHOD\quad\quad 125 \\
NOMINAL GROUP TECHNIQUE\quad\quad 125 \\
CONSENSUS CONFERENCES\quad\quad 126 \\
WIKI SYSTE\quad\quad 127 \\
DEVELOPING A NET-BASED CONSENSUS-BUILDING RESEARCH PROJECT\quad\quad 127 \\
CONCERNS WITH CONSENSUS RESEARCH\quad\quad 133 \\
TIPS FOR NET-BASED CONSENSUS RESEARCH\quad\quad 134
\newpage
PAGE 10\\
SUMMARY\quad\quad 135 \\
REFERENCES\quad\quad 135
\titulo{CHAPTER TEN} 
Quantitative Data Gathering and Analysis on the Net\quad\quad 137 \\
QUANTITATIVE STATISTICS ON INITERNET SIZE, USAGE,\\
AND DEMOGRAPHICS\quad\quad 138 \\
WEB SITE ANALYTICS OR e-METRICS\quad\quad 139 \\
WHO IS REALLY VISITING MY SITE? PROBLEMS OF PROXIES AND \\
ANONYMOUS USERS\quad\quad 142 \\
USE OF THE WEB FOR OBSERVATION OF NET-BASED ACTIVITIES\quad\quad 143\\ 
SUMMARY\quad\quad 145 \\
REFERENCES\quad\quad 145 
\titulo{CHAPTER ELEVEN}
Surveys\quad\quad 146 \\
WHY USE SURVEYS?\quad\quad 147\\
WHY USE e-SURVEYS?\quad\quad 147\\
DISADVANTAGES OF e-SURVEYS\quad\quad 150\\
CRITICAL ISSUES IN e-SURVEY DESIGN AND ADMINISTRATION\quad\quad 151 \\
ACHIEVING A HIGH RESPONSE RATE\quad\quad 153 \\
CREATING EFFECTIVE e-SURVEYITEMS\quad\quad 155 \\
CREATING AN EFFECTIVE COVER LEITER\quad\quad 156 \\
INSURING THE QUALITY OF e-SURVEYS\quad\quad 158 \\
TYPES OF e-SURVEYS\quad\quad 159 \\
TIPS FOR EMAIL SURVEYS\quad\quad 161 \\
TIPS FOR WEB-BASED SURVEYS\quad\quad 164 \\
COMMERCIAL e-SURVEY PACKAGES\quad\quad 168 \\
FFXIVRES OF POPULAR SURVEY PACKAGES\quad\quad 168 \\
WINNING COMMERCIAL e-SURVEY PRODUCTS\quad\quad 170
\end{document}